% Metódy inžinierskej práce

\documentclass[10pt,english,a4paper]{article}

%\usepackage[slovak]{babel}
%\usepackage[T1]{fontenc}
\usepackage[utf8]{inputenc}
\usepackage{graphicx}
\usepackage{url} % príkaz \url na formátovanie URL
\usepackage{hyperref} % odkazy v texte budú aktívne (pri niektorých triedach dokumentov spôsobuje posun textu)
\graphicspath{ {./images/} }
\usepackage{cite}
%\usepackage{times}

%\pagestyle{headings}

\title{Technology in Language Learning\thanks{Semestrálny projekt v predmete Metódy inžinierskej práce, ak. rok 2020/21, vedenie: Michal Hatala}} % meno a priezvisko vyučujúceho na cvičeniach

\author{Marcell Jávorka\\[2pt]
	{\small Slovak University of Technology in Bratislava}\\
	{\small Faculty of Informatics and Information Technology}\\
	{\small \texttt{xjavorkam@stuba.sk}}
	}

\date{\small 31. oktober 2020} % upravte




\begin{document}

\maketitle

\begin{abstract}
\ldots
\end{abstract}



\section{Intoduction}

The availability of new technologies has completely revolutionized teaching and learning in the past few decades. The increasing availability of internet connection has also opened many doors to people, who may not have had the opportunity to access as much information. But recently, with the role of smartphones in our lives skyrocketing, MALL (mobile-assisted language learning) has become the dominating way to learn and practice foreign languages. With portable devices tablets and mobiles, learners have full control over what, where, and how fast they learn. Distance learning has also become a possibility with students being able “attend” lectures and engage with others in a variety of ways that have not been possible before.


\section{In the past...} 


\subsection{The begining}
Since the invention of the first computers, people have tried to incorporate them in some way in education \cite {john2018evolution}. The early names associated with language learning such as CAI (computer aided instruction) and PLATO (Programmed logic for Automated Teaching Operations) show how they tried to implement machines into learning. Learners sat at the machine and pressed buttons, completing more and more difficult tasks, and a multiple-choice test would let them move on to the next chapter of learning material \cite {farr2016routledge}. CALL (computer assisted language learning) refers to technology-based language learning including CD-ROMs containing interactive multimedia and other language exercises \cite {chapelle2010spread}. Later the term was somewhat changed to TELL (Technology-enhanced language learning) to make it more fitting to its use. It brought more interaction for learners, using audio-visual tools to help the learning process. As more computer programs became available, new terminologies have emerged such as computer-assisted writing (CAW) or Computer applications in second language acquisition(CASLA) but CALL and TELL remained the most widely used term\cite {farr2016routledge}, up until the invention of the Internet.

\subsection{The internet}
The rise of the internet and with that the computer-meditated communication have reshaped not only the uses of computers, but our entire lives as well. At the end of the 20th century, this was the first-time learners could communicate inexpensively and quickly with others all over the world. Computers were now used not only for information processing, but also for communication. Tim Berners-Lee’s invention, the World Wide Web has transformed academia, business, entertainment education and so much more. It not only gave access to an unprecedented amount of information, but also allowed people to publish and share their own \cite {warschauer1998computers}. 




\section{M-Learning and MALL (mobile-assisted language learning)}

Nowadays, when everyone has some kind of smartphone and has access to the internet, we have the possibility to interact with people all over the world. The internet made online and distance learning possible from all parts of the world. The term mobility (as an origin to mobile) describes well what phones have given us. With portable and personal devices, learners can be more flexible, without the constraint on places. The features allow M-Learning to play a major role not only in language learning, but also in education as a whole.
MALL is a subdivision of both CALL and M-Learning, it deals with the use of mobile technology in language learning. Students are not restricted to learning in classroom. They have the opportunity to learn whenever and wherever they desire. \newline The most important differences between mobile learning and CALL are:

\begin{itemize}
\item Mobility
\item Social connectivity
\item Individuality
\end{itemize}

These features cannot be offered by desktop computers.
Students are not restricted to learning in a classroom, they have the opportunity to learn whenever and wherever they desire. 

Areas of mobile-based language learning are diverse, among which the most common ones are: 

\begin{enumerate}

\item Vocabulary \newline
There are many different strategies to learning vocabulary via mobile phones, each more effective than the other. Different ones are used based on the level of language proficiency of learners. They can be provided with vocabulary practices to complete and send back to their instructors, or they can be accompanied by visual and/or audio content, to help those who retain information easier that way.

\item Listening – Pronunciation - Phonetics \newline
Listening exercises also help the learner’s pronunciation tremendously. A good m-learning service should consist of speech facilities for transmitting voice. Mobile devices with multimedia function also allow the students to record their own voice, so teachers can evaluate and assess their strengths and weaknesses. This way the both the speaking skills and pronunciation can be well improved.
Also, the latest boom in the popularity of podcasts allow people to listen to a discussion about  a topic of their choice, while also practicing the language.

\item Reading comprehension \newline
Reading practices help learners not only in reading comprehension, but also expands their vocabulary. 

\item Grammar \newline
Grammatical rules are mostly taught through mobile apps, where learners can be shown some examples of the topic in hand, followed by some multiple-choice activities where they select the correct answer based on pre-given options – these can be in form of true-or-false or fill-in-the-blank which they have to solve.

\end{enumerate}


\section{Mobile applications for Language learning} 
Mobile applications, or “apps” are software designed to run on a smartphone or tablet. They serve to provide users with similar services to those accessed on PC. They usually can be downloaded from app stores such as “Google Play” for Android phones, “App Store” for devices using the IOS operating system. Some apps you have to pay for and some are free to download, which usually run ads, or have some kind of a premium subscription, that unlocks all the features of the application. \newline
Language learners can be divided into 3 different classes:

\begin{itemize}
    \item Primary Learners
    \item Secondary Learners
    \item Tertiary Learners
\end{itemize}

\subsection{Primary Learners} 

\ldots

\subsection{Secondary learners}

Secondary learners are students aged between 12-17. This group need less visual help than Primary learners, their education focuses more on vocabulary acquisition and language-use. Listening and speaking exercises can help their skills tremendously. Using language learning apps will help secondary learners to elevate their language skills to the next level. If successful, they will improve their language use, writing skills, vocabulary, grammar spelling and more.

\subsection{Tertiary Learners}

Tertiary Learners include college students and adults, who look to improve their language skills. MALL can not only enhance their language ability, but also increase their motivation. Learners can improve their listening skills, broaden their vocabulary, enhance speaking skills and pronunciation.  \cite {franklin2011mobile}\cite {howard2017any}\cite {miangah2012mobile}

%\section{One more}

%\ldots

\section{An overview of Language Learning apps}
\ldots
The reality is that the App market is like a jungle. There is too much software for students to choose and use.

\begin{center}
 \begin{tabular}{|| c | c | c | c | c ||} 
 \hline
     & Name & System & Free/Paid & Advantages  \\ 
 \hline\hline
 \includegraphics[width=2cm]{images/Duolingo logo.png} & Duolingo & Android/IOS & Free with subscription & Lorem ipsum \\ 
 \hline
 2 & X & Andorid & Free & Grammar, evaluation \\
 \hline
 3 & Y & IOS & Paid subscription & 3 \\
 \hline
 4 & Z & IOS/Android & Paid & W \\
 \hline
 5 & W & IOS & TBD & W \\
 \hline
 \end{tabular}
\end{center}

\section{Conclusion} 

\ldots



%\acknowledgement{Ak niekomu chcete poďakovať\ldots}


% týmto sa generuje zoznam literatúry z obsahu súboru literatura.bib podľa toho, na čo sa v článku odkazujete
\bibliography{literatura}
\bibliographystyle{plain} % prípadne alpha, abbrv alebo hociktorý iný
\end{document}
